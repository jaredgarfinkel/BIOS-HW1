\documentclass[]{article}
\usepackage{lmodern}
\usepackage{amssymb,amsmath}
\usepackage{ifxetex,ifluatex}
\usepackage{fixltx2e} % provides \textsubscript
\ifnum 0\ifxetex 1\fi\ifluatex 1\fi=0 % if pdftex
  \usepackage[T1]{fontenc}
  \usepackage[utf8]{inputenc}
\else % if luatex or xelatex
  \ifxetex
    \usepackage{mathspec}
  \else
    \usepackage{fontspec}
  \fi
  \defaultfontfeatures{Ligatures=TeX,Scale=MatchLowercase}
\fi
% use upquote if available, for straight quotes in verbatim environments
\IfFileExists{upquote.sty}{\usepackage{upquote}}{}
% use microtype if available
\IfFileExists{microtype.sty}{%
\usepackage{microtype}
\UseMicrotypeSet[protrusion]{basicmath} % disable protrusion for tt fonts
}{}
\usepackage[margin=1in]{geometry}
\usepackage{hyperref}
\hypersetup{unicode=true,
            pdftitle={BIOS\_HW\#1},
            pdfauthor={Jared Garfinkel},
            pdfborder={0 0 0},
            breaklinks=true}
\urlstyle{same}  % don't use monospace font for urls
\usepackage{color}
\usepackage{fancyvrb}
\newcommand{\VerbBar}{|}
\newcommand{\VERB}{\Verb[commandchars=\\\{\}]}
\DefineVerbatimEnvironment{Highlighting}{Verbatim}{commandchars=\\\{\}}
% Add ',fontsize=\small' for more characters per line
\usepackage{framed}
\definecolor{shadecolor}{RGB}{248,248,248}
\newenvironment{Shaded}{\begin{snugshade}}{\end{snugshade}}
\newcommand{\AlertTok}[1]{\textcolor[rgb]{0.94,0.16,0.16}{#1}}
\newcommand{\AnnotationTok}[1]{\textcolor[rgb]{0.56,0.35,0.01}{\textbf{\textit{#1}}}}
\newcommand{\AttributeTok}[1]{\textcolor[rgb]{0.77,0.63,0.00}{#1}}
\newcommand{\BaseNTok}[1]{\textcolor[rgb]{0.00,0.00,0.81}{#1}}
\newcommand{\BuiltInTok}[1]{#1}
\newcommand{\CharTok}[1]{\textcolor[rgb]{0.31,0.60,0.02}{#1}}
\newcommand{\CommentTok}[1]{\textcolor[rgb]{0.56,0.35,0.01}{\textit{#1}}}
\newcommand{\CommentVarTok}[1]{\textcolor[rgb]{0.56,0.35,0.01}{\textbf{\textit{#1}}}}
\newcommand{\ConstantTok}[1]{\textcolor[rgb]{0.00,0.00,0.00}{#1}}
\newcommand{\ControlFlowTok}[1]{\textcolor[rgb]{0.13,0.29,0.53}{\textbf{#1}}}
\newcommand{\DataTypeTok}[1]{\textcolor[rgb]{0.13,0.29,0.53}{#1}}
\newcommand{\DecValTok}[1]{\textcolor[rgb]{0.00,0.00,0.81}{#1}}
\newcommand{\DocumentationTok}[1]{\textcolor[rgb]{0.56,0.35,0.01}{\textbf{\textit{#1}}}}
\newcommand{\ErrorTok}[1]{\textcolor[rgb]{0.64,0.00,0.00}{\textbf{#1}}}
\newcommand{\ExtensionTok}[1]{#1}
\newcommand{\FloatTok}[1]{\textcolor[rgb]{0.00,0.00,0.81}{#1}}
\newcommand{\FunctionTok}[1]{\textcolor[rgb]{0.00,0.00,0.00}{#1}}
\newcommand{\ImportTok}[1]{#1}
\newcommand{\InformationTok}[1]{\textcolor[rgb]{0.56,0.35,0.01}{\textbf{\textit{#1}}}}
\newcommand{\KeywordTok}[1]{\textcolor[rgb]{0.13,0.29,0.53}{\textbf{#1}}}
\newcommand{\NormalTok}[1]{#1}
\newcommand{\OperatorTok}[1]{\textcolor[rgb]{0.81,0.36,0.00}{\textbf{#1}}}
\newcommand{\OtherTok}[1]{\textcolor[rgb]{0.56,0.35,0.01}{#1}}
\newcommand{\PreprocessorTok}[1]{\textcolor[rgb]{0.56,0.35,0.01}{\textit{#1}}}
\newcommand{\RegionMarkerTok}[1]{#1}
\newcommand{\SpecialCharTok}[1]{\textcolor[rgb]{0.00,0.00,0.00}{#1}}
\newcommand{\SpecialStringTok}[1]{\textcolor[rgb]{0.31,0.60,0.02}{#1}}
\newcommand{\StringTok}[1]{\textcolor[rgb]{0.31,0.60,0.02}{#1}}
\newcommand{\VariableTok}[1]{\textcolor[rgb]{0.00,0.00,0.00}{#1}}
\newcommand{\VerbatimStringTok}[1]{\textcolor[rgb]{0.31,0.60,0.02}{#1}}
\newcommand{\WarningTok}[1]{\textcolor[rgb]{0.56,0.35,0.01}{\textbf{\textit{#1}}}}
\usepackage{longtable,booktabs}
\usepackage{graphicx,grffile}
\makeatletter
\def\maxwidth{\ifdim\Gin@nat@width>\linewidth\linewidth\else\Gin@nat@width\fi}
\def\maxheight{\ifdim\Gin@nat@height>\textheight\textheight\else\Gin@nat@height\fi}
\makeatother
% Scale images if necessary, so that they will not overflow the page
% margins by default, and it is still possible to overwrite the defaults
% using explicit options in \includegraphics[width, height, ...]{}
\setkeys{Gin}{width=\maxwidth,height=\maxheight,keepaspectratio}
\IfFileExists{parskip.sty}{%
\usepackage{parskip}
}{% else
\setlength{\parindent}{0pt}
\setlength{\parskip}{6pt plus 2pt minus 1pt}
}
\setlength{\emergencystretch}{3em}  % prevent overfull lines
\providecommand{\tightlist}{%
  \setlength{\itemsep}{0pt}\setlength{\parskip}{0pt}}
\setcounter{secnumdepth}{0}
% Redefines (sub)paragraphs to behave more like sections
\ifx\paragraph\undefined\else
\let\oldparagraph\paragraph
\renewcommand{\paragraph}[1]{\oldparagraph{#1}\mbox{}}
\fi
\ifx\subparagraph\undefined\else
\let\oldsubparagraph\subparagraph
\renewcommand{\subparagraph}[1]{\oldsubparagraph{#1}\mbox{}}
\fi

%%% Use protect on footnotes to avoid problems with footnotes in titles
\let\rmarkdownfootnote\footnote%
\def\footnote{\protect\rmarkdownfootnote}

%%% Change title format to be more compact
\usepackage{titling}

% Create subtitle command for use in maketitle
\providecommand{\subtitle}[1]{
  \posttitle{
    \begin{center}\large#1\end{center}
    }
}

\setlength{\droptitle}{-2em}

  \title{BIOS\_HW\#1}
    \pretitle{\vspace{\droptitle}\centering\huge}
  \posttitle{\par}
    \author{Jared Garfinkel}
    \preauthor{\centering\large\emph}
  \postauthor{\par}
      \predate{\centering\large\emph}
  \postdate{\par}
    \date{9/13/2019}


\begin{document}
\maketitle

\hypertarget{biostatistical-methods-i}{%
\section{Biostatistical Methods I}\label{biostatistical-methods-i}}

\hypertarget{problem-1}{%
\subsection{Problem 1}\label{problem-1}}

\hypertarget{section-a}{%
\subsubsection{Section a}\label{section-a}}

\begin{verbatim}
## -- Attaching packages ------------------------------------------------------------------------------------ tidyverse 1.2.1 --
\end{verbatim}

\begin{verbatim}
## v ggplot2 3.2.1     v purrr   0.3.2
## v tibble  2.1.3     v dplyr   0.8.3
## v tidyr   0.8.3     v stringr 1.4.0
## v readr   1.3.1     v forcats 0.4.0
\end{verbatim}

\begin{verbatim}
## -- Conflicts --------------------------------------------------------------------------------------- tidyverse_conflicts() --
## x dplyr::filter() masks stats::filter()
## x dplyr::lag()    masks stats::lag()
\end{verbatim}

This is a solution for Problem 1, section a.

There is a top row of text that can be ignored for the purposes of
creating a dataframe

Change the labels.

Clean the output.

Categorize the variables.

This is a first table.

This is a solution.

\begin{Shaded}
\begin{Highlighting}[]
\KeywordTok{summary}\NormalTok{(solution1a, }\DataTypeTok{title =} \StringTok{"Demographics and Co-Morbidities"}\NormalTok{, }\DataTypeTok{labelTranslations =}\NormalTok{ my_labels,}\DataTypeTok{text =} \OtherTok{TRUE}\NormalTok{)}
\end{Highlighting}
\end{Shaded}

\begin{verbatim}
## 
## Table: Demographics and Co-Morbidities
## 
## |                    |   Intervention (N=36)   |     Control (N=36)      |
## |:-------------------|:-----------------------:|:-----------------------:|
## |Age (yrs)           |                         |                         |
## |-  Mean (SD)        |     51.500 (10.809)     |     53.583 (9.581)      |
## |-  Median (Q1, Q3)  | 51.000 (44.750, 60.250) | 55.500 (47.500, 59.250) |
## |Gender              |                         |                         |
## |-  Male             |       16 (44.4%)        |       16 (44.4%)        |
## |-  Female           |       20 (55.6%)        |       20 (55.6%)        |
## |Race                |                         |                         |
## |-  African American |       22 (61.1%)        |       31 (86.1%)        |
## |-  Hispanic         |       14 (38.9%)        |        5 (13.9%)        |
## |Hypertensive        |                         |                         |
## |-  Yes              |       16 (44.4%)        |       14 (38.9%)        |
## |-  No               |       20 (55.6%)        |       22 (61.1%)        |
## |Type 2 Diabetes     |                         |                         |
## |-  Yes              |       17 (47.2%)        |       23 (63.9%)        |
## |-  No               |       19 (52.8%)        |       13 (36.1%)        |
## |Depression          |                         |                         |
## |-  Yes              |       23 (63.9%)        |       26 (72.2%)        |
## |-  No               |       13 (36.1%)        |       10 (27.8%)        |
## |Smoker              |                         |                         |
## |-  Yes              |       31 (86.1%)        |       31 (86.1%)        |
## |-  No               |        5 (13.9%)        |        5 (13.9%)        |
\end{verbatim}

\hypertarget{section-1bi}{%
\subsection{Section 1bi}\label{section-1bi}}

This is a solution for Section 1b.

Start with the labels.

Clean the output.

Use only the variables we are looking at.

Make a table.

Create a dataframe that splits data by factor.

\begin{longtable}[]{@{}lllll@{}}
\caption{Changes in Metabolic Paramaters}\tabularnewline
\toprule
\begin{minipage}[b]{0.15\columnwidth}\raggedright
------\strut
\end{minipage} & \begin{minipage}[b]{0.20\columnwidth}\raggedright
Intervention\strut
\end{minipage} & \begin{minipage}[b]{0.18\columnwidth}\raggedright
\strut
\end{minipage} & \begin{minipage}[b]{0.18\columnwidth}\raggedright
Control\strut
\end{minipage} & \begin{minipage}[b]{0.15\columnwidth}\raggedright
\strut
\end{minipage}\tabularnewline
\midrule
\endfirsthead
\toprule
\begin{minipage}[b]{0.15\columnwidth}\raggedright
------\strut
\end{minipage} & \begin{minipage}[b]{0.20\columnwidth}\raggedright
Intervention\strut
\end{minipage} & \begin{minipage}[b]{0.18\columnwidth}\raggedright
\strut
\end{minipage} & \begin{minipage}[b]{0.18\columnwidth}\raggedright
Control\strut
\end{minipage} & \begin{minipage}[b]{0.15\columnwidth}\raggedright
\strut
\end{minipage}\tabularnewline
\midrule
\endhead
\begin{minipage}[t]{0.15\columnwidth}\raggedright
------\strut
\end{minipage} & \begin{minipage}[t]{0.20\columnwidth}\raggedright
Baseline\strut
\end{minipage} & \begin{minipage}[t]{0.18\columnwidth}\raggedright
6-months\strut
\end{minipage} & \begin{minipage}[t]{0.18\columnwidth}\raggedright
Baseline\strut
\end{minipage} & \begin{minipage}[t]{0.15\columnwidth}\raggedright
6-months\strut
\end{minipage}\tabularnewline
\begin{minipage}[t]{0.15\columnwidth}\raggedright
Systolic Blood Pressure (mm Hg)\strut
\end{minipage} & \begin{minipage}[t]{0.20\columnwidth}\raggedright
133.64 +/- 15.11\strut
\end{minipage} & \begin{minipage}[t]{0.18\columnwidth}\raggedright
125.06 +/- 15.44\strut
\end{minipage} & \begin{minipage}[t]{0.18\columnwidth}\raggedright
133.47 +/- 15.94\strut
\end{minipage} & \begin{minipage}[t]{0.15\columnwidth}\raggedright
130.14 +/- 14.35\strut
\end{minipage}\tabularnewline
\begin{minipage}[t]{0.15\columnwidth}\raggedright
------\strut
\end{minipage} & \begin{minipage}[t]{0.20\columnwidth}\raggedright
134 (121.5, 144)\strut
\end{minipage} & \begin{minipage}[t]{0.18\columnwidth}\raggedright
124 (116.75, 135)\strut
\end{minipage} & \begin{minipage}[t]{0.18\columnwidth}\raggedright
131 (122.5, 143.5)\strut
\end{minipage} & \begin{minipage}[t]{0.15\columnwidth}\raggedright
127.5 (120, 140)\strut
\end{minipage}\tabularnewline
\begin{minipage}[t]{0.15\columnwidth}\raggedright
\(\Delta\)\strut
\end{minipage} & \begin{minipage}[t]{0.20\columnwidth}\raggedright
--------\strut
\end{minipage} & \begin{minipage}[t]{0.18\columnwidth}\raggedright
-8.58 +/- 17.17\strut
\end{minipage} & \begin{minipage}[t]{0.18\columnwidth}\raggedright
--------\strut
\end{minipage} & \begin{minipage}[t]{0.15\columnwidth}\raggedright
-3.33 +/- 14.81\strut
\end{minipage}\tabularnewline
\begin{minipage}[t]{0.15\columnwidth}\raggedright
Diastolic Blood Pressure (mm Hg)\strut
\end{minipage} & \begin{minipage}[t]{0.20\columnwidth}\raggedright
75.44 +/- 9.1\strut
\end{minipage} & \begin{minipage}[t]{0.18\columnwidth}\raggedright
74.58 +/- 12.37\strut
\end{minipage} & \begin{minipage}[t]{0.18\columnwidth}\raggedright
77.14 +/- 9.66\strut
\end{minipage} & \begin{minipage}[t]{0.15\columnwidth}\raggedright
75.69 +/- 7.54\strut
\end{minipage}\tabularnewline
\begin{minipage}[t]{0.15\columnwidth}\raggedright
------\strut
\end{minipage} & \begin{minipage}[t]{0.20\columnwidth}\raggedright
74.5 (69, 81)\strut
\end{minipage} & \begin{minipage}[t]{0.18\columnwidth}\raggedright
74 (65, 80.5)\strut
\end{minipage} & \begin{minipage}[t]{0.18\columnwidth}\raggedright
76 (68.75, 85)\strut
\end{minipage} & \begin{minipage}[t]{0.15\columnwidth}\raggedright
76.5 (69, 82)\strut
\end{minipage}\tabularnewline
\begin{minipage}[t]{0.15\columnwidth}\raggedright
\(\Delta\)\strut
\end{minipage} & \begin{minipage}[t]{0.20\columnwidth}\raggedright
--------\strut
\end{minipage} & \begin{minipage}[t]{0.18\columnwidth}\raggedright
-0.86 +/- 8.3\strut
\end{minipage} & \begin{minipage}[t]{0.18\columnwidth}\raggedright
--------\strut
\end{minipage} & \begin{minipage}[t]{0.15\columnwidth}\raggedright
-1.44 +/- 10.11\strut
\end{minipage}\tabularnewline
\begin{minipage}[t]{0.15\columnwidth}\raggedright
Body Mass Index\strut
\end{minipage} & \begin{minipage}[t]{0.20\columnwidth}\raggedright
31.97 +/- 6.58\strut
\end{minipage} & \begin{minipage}[t]{0.18\columnwidth}\raggedright
31.21 +/- 6.13\strut
\end{minipage} & \begin{minipage}[t]{0.18\columnwidth}\raggedright
34.23 +/- 6.16\strut
\end{minipage} & \begin{minipage}[t]{0.15\columnwidth}\raggedright
34.51 +/- 5.97\strut
\end{minipage}\tabularnewline
\begin{minipage}[t]{0.15\columnwidth}\raggedright
------\strut
\end{minipage} & \begin{minipage}[t]{0.20\columnwidth}\raggedright
29.25 (27.375, 34.1)\strut
\end{minipage} & \begin{minipage}[t]{0.18\columnwidth}\raggedright
29.15 (26.8, 32.875)\strut
\end{minipage} & \begin{minipage}[t]{0.18\columnwidth}\raggedright
33.4 (29.6, 37.575)\strut
\end{minipage} & \begin{minipage}[t]{0.15\columnwidth}\raggedright
33.05 (30.425, 37.55)\strut
\end{minipage}\tabularnewline
\begin{minipage}[t]{0.15\columnwidth}\raggedright
\(\Delta\)\strut
\end{minipage} & \begin{minipage}[t]{0.20\columnwidth}\raggedright
--------\strut
\end{minipage} & \begin{minipage}[t]{0.18\columnwidth}\raggedright
-0.76 +/- 1.44\strut
\end{minipage} & \begin{minipage}[t]{0.18\columnwidth}\raggedright
--------\strut
\end{minipage} & \begin{minipage}[t]{0.15\columnwidth}\raggedright
0.28 +/- 0.97\strut
\end{minipage}\tabularnewline
\begin{minipage}[t]{0.15\columnwidth}\raggedright
HDL Cholesterol (mg/dL)\strut
\end{minipage} & \begin{minipage}[t]{0.20\columnwidth}\raggedright
50.17 +/- 11.85\strut
\end{minipage} & \begin{minipage}[t]{0.18\columnwidth}\raggedright
50.17 +/- 13.07\strut
\end{minipage} & \begin{minipage}[t]{0.18\columnwidth}\raggedright
48.33 +/- 13.7\strut
\end{minipage} & \begin{minipage}[t]{0.15\columnwidth}\raggedright
45.19 +/- 10.78\strut
\end{minipage}\tabularnewline
\begin{minipage}[t]{0.15\columnwidth}\raggedright
------\strut
\end{minipage} & \begin{minipage}[t]{0.20\columnwidth}\raggedright
47.5 (40, 60)\strut
\end{minipage} & \begin{minipage}[t]{0.18\columnwidth}\raggedright
48.5 (43, 60.25)\strut
\end{minipage} & \begin{minipage}[t]{0.18\columnwidth}\raggedright
43.5 (39, 54.25)\strut
\end{minipage} & \begin{minipage}[t]{0.15\columnwidth}\raggedright
43.5 (38, 52)\strut
\end{minipage}\tabularnewline
\begin{minipage}[t]{0.15\columnwidth}\raggedright
\(\Delta\)\strut
\end{minipage} & \begin{minipage}[t]{0.20\columnwidth}\raggedright
--------\strut
\end{minipage} & \begin{minipage}[t]{0.18\columnwidth}\raggedright
0 +/- 8.09\strut
\end{minipage} & \begin{minipage}[t]{0.18\columnwidth}\raggedright
--------\strut
\end{minipage} & \begin{minipage}[t]{0.15\columnwidth}\raggedright
-3.14 +/- 6.91\strut
\end{minipage}\tabularnewline
\begin{minipage}[t]{0.15\columnwidth}\raggedright
LDL Cholesterol (mg/dL)\strut
\end{minipage} & \begin{minipage}[t]{0.20\columnwidth}\raggedright
102.94 +/- 33.84\strut
\end{minipage} & \begin{minipage}[t]{0.18\columnwidth}\raggedright
100.5 +/- 30.39\strut
\end{minipage} & \begin{minipage}[t]{0.18\columnwidth}\raggedright
99.83 +/- 29.06\strut
\end{minipage} & \begin{minipage}[t]{0.15\columnwidth}\raggedright
93.61 +/- 27.47\strut
\end{minipage}\tabularnewline
\begin{minipage}[t]{0.15\columnwidth}\raggedright
------\strut
\end{minipage} & \begin{minipage}[t]{0.20\columnwidth}\raggedright
109 (75.25, 124.5)\strut
\end{minipage} & \begin{minipage}[t]{0.18\columnwidth}\raggedright
95 (76.5, 120.5)\strut
\end{minipage} & \begin{minipage}[t]{0.18\columnwidth}\raggedright
104 (88.25, 112.25)\strut
\end{minipage} & \begin{minipage}[t]{0.15\columnwidth}\raggedright
96.5 (77.5, 110.25)\strut
\end{minipage}\tabularnewline
\begin{minipage}[t]{0.15\columnwidth}\raggedright
\(\Delta\)\strut
\end{minipage} & \begin{minipage}[t]{0.20\columnwidth}\raggedright
--------\strut
\end{minipage} & \begin{minipage}[t]{0.18\columnwidth}\raggedright
-2.44 +/- 21.27\strut
\end{minipage} & \begin{minipage}[t]{0.18\columnwidth}\raggedright
--------\strut
\end{minipage} & \begin{minipage}[t]{0.15\columnwidth}\raggedright
-6.22 +/- 23.12\strut
\end{minipage}\tabularnewline
\begin{minipage}[t]{0.15\columnwidth}\raggedright
Blood Glucose Level (mmol/L)\strut
\end{minipage} & \begin{minipage}[t]{0.20\columnwidth}\raggedright
116.64 +/- 74.91\strut
\end{minipage} & \begin{minipage}[t]{0.18\columnwidth}\raggedright
107.14 +/- 38.65\strut
\end{minipage} & \begin{minipage}[t]{0.18\columnwidth}\raggedright
128.97 +/- 73.86\strut
\end{minipage} & \begin{minipage}[t]{0.15\columnwidth}\raggedright
126.61 +/- 63.96\strut
\end{minipage}\tabularnewline
\begin{minipage}[t]{0.15\columnwidth}\raggedright
------\strut
\end{minipage} & \begin{minipage}[t]{0.20\columnwidth}\raggedright
94 (83.75, 116.5)\strut
\end{minipage} & \begin{minipage}[t]{0.18\columnwidth}\raggedright
95.5 (85.25, 129)\strut
\end{minipage} & \begin{minipage}[t]{0.18\columnwidth}\raggedright
98 (81.75, 139)\strut
\end{minipage} & \begin{minipage}[t]{0.15\columnwidth}\raggedright
106.5 (85, 145.75)\strut
\end{minipage}\tabularnewline
\begin{minipage}[t]{0.15\columnwidth}\raggedright
\(\Delta\)\strut
\end{minipage} & \begin{minipage}[t]{0.20\columnwidth}\raggedright
--------\strut
\end{minipage} & \begin{minipage}[t]{0.18\columnwidth}\raggedright
-9.5 +/- 57.36\strut
\end{minipage} & \begin{minipage}[t]{0.18\columnwidth}\raggedright
--------\strut
\end{minipage} & \begin{minipage}[t]{0.15\columnwidth}\raggedright
-2.36 +/- 51.22\strut
\end{minipage}\tabularnewline
\bottomrule
\end{longtable}

\hypertarget{problem-1bii}{%
\section{Problem 1bii}\label{problem-1bii}}

Two boxplots of intervention groups.

\begin{Shaded}
\begin{Highlighting}[]
\KeywordTok{boxplot}\NormalTok{(BMI_good,}
        \DataTypeTok{names =}\NormalTok{ BMI_labels,}
        \DataTypeTok{xlab =} \StringTok{"Intervention vs. Control"}\NormalTok{,}
        \DataTypeTok{ylab =} \StringTok{"Body Mass Index"}\NormalTok{, }
        \DataTypeTok{col =} \KeywordTok{c}\NormalTok{(}\StringTok{"blue"}\NormalTok{,}\StringTok{"red"}\NormalTok{), }
        \DataTypeTok{main =} \StringTok{"Effects of exercise on BMI"}\NormalTok{)}
\end{Highlighting}
\end{Shaded}

\includegraphics{BIOS-HW1_files/figure-latex/split a big chunk BMI+LDL-1.pdf}

\begin{Shaded}
\begin{Highlighting}[]
\KeywordTok{boxplot}\NormalTok{(LDL_good,}
        \DataTypeTok{names =}\NormalTok{ LDL_labels,}
        \DataTypeTok{xlab =} \StringTok{"Intervention vs. Control"}\NormalTok{,}
        \DataTypeTok{ylab =} \StringTok{"LDL Cholesterol (mg/dL)"}\NormalTok{, }
        \DataTypeTok{col =} \KeywordTok{c}\NormalTok{(}\StringTok{"blue"}\NormalTok{,}\StringTok{"red"}\NormalTok{), }
        \DataTypeTok{main =} \StringTok{"Effects of exercise on LDL"}\NormalTok{)}
\end{Highlighting}
\end{Shaded}

\includegraphics{BIOS-HW1_files/figure-latex/split a big chunk BMI+LDL-2.pdf}

\hypertarget{section-1biii}{%
\subsection{Section 1biii}\label{section-1biii}}

These findings indicate that among the intervention and control

groups there were no changes in BMI and LDL. This might be because

of the matching technique used, where participants and non-participants

were matched by gender. A better method may be to match the participants

and non-participants by smoking status, or other health-related

indicators.

\hypertarget{problem-2}{%
\section{Problem 2}\label{problem-2}}

The probability that a triple test of a fetus with down syndrome is
positive

is P(+\textbar{}DS) = 0.60.

The probability that a triple test of a fetus that does not have down
syndrome is positive

is P(+\textbar{}NDS) = 0.05.

The probability of a fetus being born with Down Syndrom is P(DS) = 0.001

What is the probability that a fetus with a positive test result
actually has Down Syndrome? P(DS\textbar{}+) = ?

P(DS\textbar{}+) = \[P(+|DS)*P(DS) / (P(+|DS)*P(DS)+P(+|NDS)*P(NDS))\]

= \[0.60*0.001 / (0.60*0.001+0.05*(1-0.001))\]

= 0.01

This is called a positive predictive value. Because the positive
predictive value

of this test is less than two percent, it is very unlikely to be
accurate.

This might mean doing the test two, three, or more times to achieve a
reasonable

level of certainty regarding the presence of the condition.

\hypertarget{problem-3}{%
\section{Problem 3}\label{problem-3}}

The media reported widely on a story about an article titled, ``Habitual
tea drinking modulates brain efficiency: evidence from brain
connectivity evaluation'' published in the journal, ``Aging'' based in
Albany, NY. The study used neuroimaging from subjects split into two
groups, ``tea drinkers'' and ``non-tea drinkers'' to analyze differences
in brain structure. Reporting on the findings had some similarities. For
instance, all the reporting in this analysis promoted the benefits of
drinking tea for healthy aging and brain structure. Some of the media
included descriptions of the methods, like the number of participants
and the thing being studied. There were also references to brain
structure in all the articles, which was the main topic of the paper.
One reporter claimed the study took place over three years.

I think the reporting was fairly measured across the board. This is
reasonable considering it is an exploratory analysis. One thing that I
found interesting is that the authors used leftward assymetry to measure
the amount of aging on a brain. One of the main findings of the paper
was that there was more leftward assymetry in the non-tea drinking
group. The study was meant to add to the literature suggesting a
correlation between drinking tea and healthier aging. I think the
authors accomplished this even though the study size was small (n=36)
and the results were mixed. While functional brain networks were found
to be more efficient among ``tea drinkers'', structural brain networks
were not.

While this is an early look at the brain structure of tea drinkers
compared to non-tea drinkers, it follows a large amount of literature
suggesting the health benefits of drinking tea. In fact, the authors of
this study conducted a longitudinal study of tea drinkers in 2017 that
had encouraging results. There have also been press releases from
Harvard University promoting the health benefits of tea on several
health factors. In the future, studies like this may determine the
mechanisms by which tea can bolster health. Until then, I think it is
safe to continue to drink tea, which may even impart a benefit.

\hypertarget{sources}{%
\subsubsection{Sources:}\label{sources}}

\hypertarget{httpswww.mindbodygreen.comarticlesnew-study-shows-regular-tea-drinkers-have-better-organized-brain-regionsfbclidiwar34q8crryfvod61zpoolk803psg88ponqdedgr8l_jy9cqn8ce-97_dvwq}{%
\paragraph{\texorpdfstring{1.
\url{https://www.mindbodygreen.com/articles/new-study-shows-regular-tea-drinkers-have-better-organized-brain-regions?fbclid=IwAR34Q8CrrYFvod61zpOolK803psG88POnQdeDGr8l_jY9CQN8CE-97_DVWQ}}{1. https://www.mindbodygreen.com/articles/new-study-shows-regular-tea-drinkers-have-better-organized-brain-regions?fbclid=IwAR34Q8CrrYFvod61zpOolK803psG88POnQdeDGr8l\_jY9CQN8CE-97\_DVWQ}}\label{httpswww.mindbodygreen.comarticlesnew-study-shows-regular-tea-drinkers-have-better-organized-brain-regionsfbclidiwar34q8crryfvod61zpoolk803psg88ponqdedgr8l_jy9cqn8ce-97_dvwq}}

\hypertarget{httpswww.sciencedaily.comreleases201909190912100945.htm}{%
\paragraph{\texorpdfstring{2.
\url{https://www.sciencedaily.com/releases/2019/09/190912100945.htm}}{2. https://www.sciencedaily.com/releases/2019/09/190912100945.htm}}\label{httpswww.sciencedaily.comreleases201909190912100945.htm}}

\hypertarget{httpswww.businessinsider.sgdrinking-tea-could-save-your-brain-from-old-age-decline-nus-led-study-finds}{%
\paragraph{\texorpdfstring{3.
\url{https://www.businessinsider.sg/drinking-tea-could-save-your-brain-from-old-age-decline-nus-led-study-finds/}}{3. https://www.businessinsider.sg/drinking-tea-could-save-your-brain-from-old-age-decline-nus-led-study-finds/}}\label{httpswww.businessinsider.sgdrinking-tea-could-save-your-brain-from-old-age-decline-nus-led-study-finds}}

\hypertarget{httpsneurosciencenews.comtea-drinking-brain-health-14889}{%
\paragraph{\texorpdfstring{4.
\url{https://neurosciencenews.com/tea-drinking-brain-health-14889/}}{4. https://neurosciencenews.com/tea-drinking-brain-health-14889/}}\label{httpsneurosciencenews.comtea-drinking-brain-health-14889}}

\hypertarget{httpswww.independent.co.uklife-stylehealth-and-familiestea-brain-health-boost-ageing-green-black-oolong-a9107546.html}{%
\paragraph{\texorpdfstring{5.
\url{https://www.independent.co.uk/life-style/health-and-families/tea-brain-health-boost-ageing-green-black-oolong-a9107546.html}}{5. https://www.independent.co.uk/life-style/health-and-families/tea-brain-health-boost-ageing-green-black-oolong-a9107546.html}}\label{httpswww.independent.co.uklife-stylehealth-and-familiestea-brain-health-boost-ageing-green-black-oolong-a9107546.html}}

\hypertarget{httpswww.aging-us.comarticle102023text}{%
\paragraph{\texorpdfstring{6.
\url{https://www.aging-us.com/article/102023/text}}{6. https://www.aging-us.com/article/102023/text}}\label{httpswww.aging-us.comarticle102023text}}


\end{document}
